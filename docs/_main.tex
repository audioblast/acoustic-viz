% Options for packages loaded elsewhere
\PassOptionsToPackage{unicode}{hyperref}
\PassOptionsToPackage{hyphens}{url}
%
\documentclass[
]{book}
\usepackage{amsmath,amssymb}
\usepackage{lmodern}
\usepackage{iftex}
\ifPDFTeX
  \usepackage[T1]{fontenc}
  \usepackage[utf8]{inputenc}
  \usepackage{textcomp} % provide euro and other symbols
\else % if luatex or xetex
  \usepackage{unicode-math}
  \defaultfontfeatures{Scale=MatchLowercase}
  \defaultfontfeatures[\rmfamily]{Ligatures=TeX,Scale=1}
\fi
% Use upquote if available, for straight quotes in verbatim environments
\IfFileExists{upquote.sty}{\usepackage{upquote}}{}
\IfFileExists{microtype.sty}{% use microtype if available
  \usepackage[]{microtype}
  \UseMicrotypeSet[protrusion]{basicmath} % disable protrusion for tt fonts
}{}
\makeatletter
\@ifundefined{KOMAClassName}{% if non-KOMA class
  \IfFileExists{parskip.sty}{%
    \usepackage{parskip}
  }{% else
    \setlength{\parindent}{0pt}
    \setlength{\parskip}{6pt plus 2pt minus 1pt}}
}{% if KOMA class
  \KOMAoptions{parskip=half}}
\makeatother
\usepackage{xcolor}
\usepackage{longtable,booktabs,array}
\usepackage{calc} % for calculating minipage widths
% Correct order of tables after \paragraph or \subparagraph
\usepackage{etoolbox}
\makeatletter
\patchcmd\longtable{\par}{\if@noskipsec\mbox{}\fi\par}{}{}
\makeatother
% Allow footnotes in longtable head/foot
\IfFileExists{footnotehyper.sty}{\usepackage{footnotehyper}}{\usepackage{footnote}}
\makesavenoteenv{longtable}
\usepackage{graphicx}
\makeatletter
\def\maxwidth{\ifdim\Gin@nat@width>\linewidth\linewidth\else\Gin@nat@width\fi}
\def\maxheight{\ifdim\Gin@nat@height>\textheight\textheight\else\Gin@nat@height\fi}
\makeatother
% Scale images if necessary, so that they will not overflow the page
% margins by default, and it is still possible to overwrite the defaults
% using explicit options in \includegraphics[width, height, ...]{}
\setkeys{Gin}{width=\maxwidth,height=\maxheight,keepaspectratio}
% Set default figure placement to htbp
\makeatletter
\def\fps@figure{htbp}
\makeatother
\setlength{\emergencystretch}{3em} % prevent overfull lines
\providecommand{\tightlist}{%
  \setlength{\itemsep}{0pt}\setlength{\parskip}{0pt}}
\setcounter{secnumdepth}{5}
\usepackage{booktabs}
\ifLuaTeX
  \usepackage{selnolig}  % disable illegal ligatures
\fi
\usepackage[]{natbib}
\bibliographystyle{plainnat}
\IfFileExists{bookmark.sty}{\usepackage{bookmark}}{\usepackage{hyperref}}
\IfFileExists{xurl.sty}{\usepackage{xurl}}{} % add URL line breaks if available
\urlstyle{same} % disable monospaced font for URLs
\hypersetup{
  pdftitle={Visualisation for bioacoustics and ecoacoustics},
  pdfauthor={Ed Baker},
  hidelinks,
  pdfcreator={LaTeX via pandoc}}

\title{Visualisation for bioacoustics and ecoacoustics}
\author{Ed Baker}
\date{2022-08-12}

\begin{document}
\maketitle

{
\setcounter{tocdepth}{1}
\tableofcontents
}
\hypertarget{about}{%
\chapter*{About}\label{about}}
\addcontentsline{toc}{chapter}{About}

Bioacoustics and ecoacoustics are rapidly advancing multi-disciplinary fields of study that focus on how organisms communicate using sound, and the overall sound of a landscape (the soundscape). Despite the focus on sound, much of the communication of ideas, and even sounds, between researchers is done using graphical representations.

This should not come as a surprise, the printing press came centuries before the radio as a means for long distance communication, and ink on paper has a permanence that sounds would not achieve for a long time after the invention of writing. The current flourishing of these disciplines is driven as much by the low cost and ease of use of products such as AudioMoth and the decreasing cost of digitial storage and processing as by novel ideas.

Visualizations of acoustic data however are not going away - we are a predominantly visual species, and as ways of summarising acoustic data - or making the ultrasound tangible, they are powerful tools in the hands of the acoustician.

\hypertarget{intro}{%
\chapter{Introduction}\label{intro}}

\hypertarget{history}{%
\chapter{History}\label{history}}

\hypertarget{musical}{%
\section{Musical}\label{musical}}

\hypertarget{descriptive}{%
\section{Descriptive}\label{descriptive}}

\hypertarget{early-viz}{%
\chapter{Early visualisations}\label{early-viz}}

\hypertarget{crts-and-print-outs.}{%
\section{CRTs and print outs.}\label{crts-and-print-outs.}}

  \bibliography{book.bib,packages.bib}

\end{document}
